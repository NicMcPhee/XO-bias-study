% Research Paper for MICS 2015
% by David Donatucci, Kirbie Dramdahl, and Nic McPhee

\documentclass[12pt]{article}

\setlength{\oddsidemargin}{0in}
\setlength{\evensidemargin}{0in}
\setlength{\topmargin}{0in}
\setlength{\headheight}{0in}
\setlength{\headsep}{0in}
\setlength{\textwidth}{6in}
\setlength{\textheight}{9in}
\setlength{\parindent}{0in} 

\usepackage{parskip}
\usepackage{times} %For typeface
\usepackage{graphicx}
\usepackage{algorithm}
\usepackage{algorithm,algorithmic}
\usepackage[justification=centering]{caption}[2007/12/23]
\usepackage{url}
\sloppy

\usepackage{float}
\newfloat{Query}{tbp}{lop}

\newcommand{\inset}[1]{$\in \{ {#1} \}$}

\newcommand{\citep}[1]{\cite{#1}}
\newcommand{\PPLR}[1]{$\eta_M$}
\newcommand{\LLR}[1]{$\eta_L$}

\DeclareGraphicsRule{.tif}{png}{.png}{`convert #1 `dirname #1`/`basename #1 .tif`.png}

\title{[Insert Title]}

\author{
 		David Donatucci, M. Kirbie Dramdahl, and Nicholas Freitag McPhee\\
        Division of Science and Mathematics\\
        University of Minnesota, Morris\\
        Morris, MN 56267\\
        donat056@morris.umn.edu\\
        dramd002@morris.umn.edu\\
        mcphee@morris.umn.edu\\
}
\date{} 

\begin{document}
\pagestyle{plain}

\maketitle

\begin{abstract}

\begin{itemize}
	\item Explain the idea of XO bias, where it came from and why
	\item XO bias rates
	\item Does seem to improve performance in certain circumstances
	\item Doesn't always improve things
	\item Probably depends on things like tournament size and elitism and pop size
		\item May depend the difference in parental fitness
	\item May increase selection pressure and (premature) convergence
	\item Applied to many problems of different types
\end{itemize}

\end{abstract}

\section{Introduction} \label{Introduction}

\section{Genetic Programming} \label{Genetic Programming}

\section{Graph Databases} \label{Graph Databases}

\section{Experimental Setup} \label{Experiments}

\subsection{Genetic Programming Setup} \label{Genetic Programming Setup}

\subsection{Neo4j Setup} \label{Neo4j Setup}

\section{Results} \label{Results}

\subsection{Number of Initial Individuals With Final Generation Descendants} \label{Number Initial Individuals With Descendants}

\subsection{Effectiveness of Mutation and Crossover} \label{Effectiveness Mutation Crossover}

\subsection{Winning Root Ancestry Line Fitness} \label{Winning Root Line Fitness}

\subsection{Most Recent Common Ancestor} \label{Most Recent Common Ancestor}

\section{Conclusions} \label{Conclusions}

\section*{Acknowledgements}

\pagebreak

\bibliographystyle{acm}
\bibliography{Research_2015}

\end{document}